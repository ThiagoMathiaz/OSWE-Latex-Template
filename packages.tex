\usepackage{lastpage}
\usepackage{tabularx}
\usepackage{cprotect}
%Seitenränder definieren
\usepackage[right=4.0cm,left=3.3cm, bottom=3.9cm, top=4.1cm, footskip=2.1cm, headsep=2.0cm]{geometry}
\usepackage[utf8]{inputenc}
\usepackage[T1]{fontenc}
\usepackage{palatino}
\linespread{1.25}
\usepackage{microtype}
\usepackage[english]{babel}
%More citations
\usepackage{natbib}
%better urls mit \url{}
\usepackage[hyphens]{url}
\usepackage{lastpage}
%code
\usepackage{listings}

%more colors
\usepackage{soul}

%Force figure to be placed “HERE”
\usepackage{float} 

%folgende Zeilen sind für Kapitelüberschriften
\usepackage[rigidchapters]{titlesec}
\usepackage{blindtext}
\titleformat{\chapter}
{\normalfont\LARGE}
{\makebox[3pc][l]{\LARGE\thechapter\hfil\rule[-6pt]{0.5pt}{2pc}}}
{0pt}
{\LARGE}
\titlespacing*{\chapter}{0pt}{0pt}{82pt}

%Csv -> Latex
\usepackage{csvsimple}

%Für Graphiken 
\usepackage{tikz}
\usetikzlibrary{plotmarks}
\usetikzlibrary{positioning,shapes,shadows,arrows}
\usepackage{graphicx}

%Paket gibt einige Optionen mehr bei Tabellen (wird eher nicht verwendet)
\usepackage{array}

%\usepackage{stdpage}
%test wegen anzahl zeilen pro seite
%Paket für Zeilenabstände
\usepackage{setspace}
%\onehalfspacing
\usepackage{multirow}
%Paket gibt mehr Kontrolle über die Captions (Bildunterschriften) bei Abbildungen
\usepackage[labelfont=bf,format=hang,font=footnotesize,justification=raggedright,singlelinecheck=false]{caption}

%Helvetia (Arial) Verwenden WICHTIG: Beide folgenden Zeilen kopieren!
%\usepackage[scaled]{helvet} %
%\renewcommand*\familydefault{\sfdefault} %%

% Abschalten des Einrückens bei neuen Absätzen (manuell, nach Tabellen, Abbildungen, etc.)
\setlength{\parindent}{0pt}
\hyphenation{}

% New Page before section
\newcommand{\sectionbreak}{\clearpage}

\usepackage{color}
\usepackage{lstautogobble}  % Fix relative indenting
\usepackage{zi4}  

\definecolor{color0}{rgb}{0,0,0}% black
\definecolor{color1}{rgb}{0.22,0.45,0.70}% light blue
\definecolor{color2}{rgb}{0.45,0.45,0.45}% dark grey
\definecolor{mygreen}{rgb}{0,0.6,0}
\definecolor{mygray}{rgb}{0.5,0.5,0.5}
\definecolor{myblue}{RGB}{40,58,93}
\definecolor{codebackground}{RGB}{221, 222, 223}
\definecolor{codechanged}{rgb}{0.8, 0.0, 0.0}


\definecolor{bluekeywords}{rgb}{0.13, 0.13, 1}
\definecolor{greencomments}{rgb}{0, 0.5, 0}
\definecolor{redstrings}{rgb}{0.9, 0, 0}
\definecolor{graynumbers}{rgb}{0.5, 0.5, 0.5}
    
\lstset{
    autogobble,
    columns=fullflexible,
    showspaces=false,
    showtabs=false,
    breaklines=true,
    showstringspaces=false,
    breakatwhitespace=true,
    commentstyle=\color{greencomments},
    keywordstyle=\color{bluekeywords},
    stringstyle=\color{redstrings},
    numberstyle=\color{graynumbers},
    basicstyle=\ttfamily\footnotesize,
    frame=l,
    framesep=12pt,
    xleftmargin=12pt,
    tabsize=4,
    captionpos=b
}

%Aktives Inhaltsverzeichnis und links
\usepackage{hyperref}
\hypersetup{
    colorlinks,
    citecolor=black,
    filecolor=black,
    linkcolor=black,
    urlcolor=black
}

\usepackage{fancyhdr}
\pagestyle{fancy}
\fancypagestyle{plain}{}
\fancyhf{}
\fancyfoot{} % clear all footer fields

\fancyhead[R]{\small{\leftmark}}
\fancyhead[L]{\small{Offensive Security - Penetration Test Report}}
\renewcommand{\sectionmark}[1]{\markboth{#1}{}}

\renewcommand{\footrulewidth}{0.1pt} % Create a rule above the page number
\fancyfoot[R]{\textcolor{color1}\thepage  \textcolor{color2}{/\pageref{LastPage}}}